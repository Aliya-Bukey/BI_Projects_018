
\documentclass[sigconf]{acmart}

\AtBeginDocument{\providecommand\BibTeX{Bib\TeX}}
\setcopyright{acmlicensed}
\copyrightyear{2025}
\acmYear{2025}
\acmDOI{XXXXXXX.XXXXXXX}

\acmConference[BI 2025]{Business Intelligence}{-}{-}

\begin{document}

\title{BI2025 Experiment Report - Group 18: Airline Passenger Satisfaction Prediction}


          \author{Aliya Bokey}
          \authornote{Modeling, Deployment, Matr.Nr.: 01234567}
          \affiliation{
            \institution{TU Wien}
            \country{Austria}
          }
          
          \author{Aliya Bokey}
          \authornote{Modeling, Deployment, Matr.Nr.: 12451104}
          \affiliation{
            \institution{TU Wien}
            \country{Austria}
          }
          
          \author{Alicya Novita Hariyanto}
          \authornote{Data Preparation, Evaluation, Matr.Nr.: 76543210}
          \affiliation{
            \institution{TU Wien}
            \country{Austria}
          }
          
          \author{Alicya Novita Hariyanto}
          \authornote{Data Preparation, Evaluation, Matr.Nr.: 12536814}
          \affiliation{
            \institution{TU Wien}
            \country{Austria}
          }
          

\begin{abstract}
This report documents a comprehensive machine learning experiment for predicting airline passenger
satisfaction, following the CRISP-DM process model.
\end{abstract}

\ccsdesc[500]{Computing methodologies~Machine learning}
\keywords{CRISP-DM, Provenance, Knowledge Graph, Machine Learning, Random Forest, Bias Evaluation}

\maketitle

\section{Business Understanding}
\subsection{Data Source and Scenario}
The dataset “Airline Passenger Satisfaction” originates from a real-world customer satisfaction survey conducted by an airline company. It contains responses from approximately 129,000 passengers and captures demographic information, travel characteristics, and detailed service quality ratings across multiple service dimensions (e.g., inflight service, seat comfort, online boarding).
The business scenario considered in this assignment is that of a commercial airline aiming to improve customer satisfaction and reduce customer churn by identifying passengers who are likely to be dissatisfied with their flight experience.

\subsection{Business Objectives}
The primary business objective is to proactively identify passengers who are likely to be dissatisfied in order to enable targeted service improvements, customer retention strategies, and operational adjustments.
A secondary objective is to gain insights into which service attributes have the strongest impact on passenger dissatisfaction.

\subsection{Business Success Criteria}
The business objective is considered successful if a predictive model can reliably identify dissatisfied passengers with sufficiently high recall, allowing the airline to intervene before dissatisfaction results in loss of customer loyalty or negative reputation effects.

\subsection{Data Mining Goals}
The data mining goal is to build a supervised classification model that predicts whether a passenger is “neutral or dissatisfied” based on demographic attributes, travel characteristics, and service quality ratings.

\subsection{Data Mining Success Criteria}
From a data mining perspective, success is defined as achieving performance significantly above a random or trivial baseline, measured using metrics such as accuracy, precision, recall, F1-score, and ROC-AUC, with particular emphasis on the minority or business-critical class of dissatisfied passengers.

\subsection{AI Risk Aspects}
Potential AI risks include bias against specific passenger groups, such as younger or older passengers, economy-class travelers, or customers traveling for personal reasons.
Additionally, the use of demographic attributes (e.g., age, gender) may raise fairness concerns if the model’s predictions disproportionately affect certain subgroups.
There is also a risk that satisfaction labels reflect subjective perceptions influenced by external factors not captured in the dataset.

\section{Data Understanding}
\subsection{Dataset Description}
Customer satisfaction survey data from 129,880 airline passengers with 24 attributes including demographic information, travel characteristics, service ratings (1-5 Likert scale), delay information, and satisfaction target.

\subsection{Feature Overview}
The dataset contains the following key features (first 10 shown):

\begin{table}[h]
  \caption{Selected Dataset Features}
  \label{tab:features}
  \begin{tabular}{lp{0.15\linewidth}p{0.35\linewidth}}
    \toprule
    \textbf{Feature Name} & \textbf{Data Type} & \textbf{Description} \\
    \midrule
    Age & integer> & Passenger age in years \\
    Arrival Delay in Minutes & float> & Arrival delay in minutes (may contain missing values) \\
    Baggage handling & integer> & Baggage handling service (1-5) \\
    Checkin service & integer> & Check-in service quality (1-5) \\
    Class & string> & Travel class: Eco, Eco Plus, or Business \\
    Cleanliness & integer> & Cleanliness of aircraft (1-5) \\
    Customer Type & string> & Loyal Customer or disloyal Customer \\
    Departure Delay in Minutes & integer> & Departure delay in minutes \\
    Departure/Arrival time convenient & integer> & Convenience of departure/arrival times (1-5) \\
    Ease of Online booking & integer> & Ease of online booking process (1-5) \\
    \bottomrule
  \end{tabular}
\end{table}

\subsection{Outlier Analysis}
Outlier analysis identified extreme values in flight distance, age, and departure delays, with appropriate handling decisions documented.

\section{Data Preparation}
\subsection{Data Cleaning and Transformation}
Data preparation involved handling missing values, capping extreme delays at 24 hours, and creating derived features such as delay categories.

\subsection{Categorical Encoding}
Encoding categorical variables for machine learning:
1. Binary variables (Gender, Customer Type): Label encoding (0/1)
   - Gender: Male=0, Female=1
   - Customer Type: Loyal Customer=0, disloyal Customer=1
2. Multi-class variables: One-hot encoding
   - Type of Travel: 2 categories → 2 binary columns
   - Class: 3 categories → 3 binary columns  
   - delay\_category: 5 categories → 5 binary columns
3. Target variable (satisfaction): Binary encoding
   - satisfied=1, neutral or dissatisfied=0
4. Original categorical columns removed to avoid redundancy

\section{Modeling}
\subsection{Algorithm Selection}
\textbf{Selected Algorithm:} Random Forest Classifier

Ensemble of decision trees using bootstrap aggregating

\subsection{Hyperparameter Tuning}
The model was tuned with the following hyperparameter configuration:

\begin{table}[h]
  \caption{Hyperparameter Settings}
  \label{tab:hyperparams}
  \begin{tabular}{lp{0.4\linewidth}l}
    \toprule
    \textbf{Parameter} & \textbf{Description} & \textbf{Value/Range} \\
    \midrule
    n\_estimators & Number of trees & [50, 100, 150, 200, 250, 300] \\
    n\_estimators & Number of trees & [50, 100, 150, 200, 250, 300] \\
    \bottomrule
  \end{tabular}
\end{table}

\subsection{Training Execution}
\begin{itemize}
    \item \textbf{Algorithm:} Random Forest Classifier
    \item \textbf{Training Samples:} 103,904 (80\% of total data)
    \item \textbf{Test Samples:} 25,976 (20\% holdout)
    \item \textbf{Training Duration:} 2026-01-11 13:39:11 to 2026-01-11 13:39:11
    \item \textbf{Final Model:} Random Forest with optimal n\_estimators
\end{itemize}

\section{Evaluation}
\subsection{Performance Metrics}
The final model achieved the following performance on the test set:

\begin{table}[h]
\centering
\begin{tabular}{lr}
\toprule
Metric & Value \\
\midrule
Accuracy  & 1.0000 \\
Precision & 1.0000 \\
Recall    & 1.0000 \\
F1-Score  & 1.0000 \\
ROC-AUC   & 1.0000 \\
\bottomrule
\end{tabular}
\caption{Model Performance Metrics}
\end{table}

\textbf{Note:} Perfect metrics (100\%) may indicate potential overfitting and should be interpreted with caution.

\subsection{Bias and Fairness Evaluation}
"""
Bias Evaluation using Gender as Protected Attribute:

Results:
- Test set: 12,819 male, 13,157 female
- Male Accuracy: 1.0000
- Female Accuracy: 1.0000
- Demographic Parity Difference: 0.0124
- Equal Opportunity Difference: 0.0000
- Bias Level: LOW

Note: 100\% accuracy metrics require caution.
"...

\subsection{Comparison with Baselines}
\begin{itemize}
    \item \textbf{Trivial Baseline:} 56.6\% accuracy (always predict majority class)
    \item \textbf{Random Baseline:} 50\% accuracy
    \item \textbf{Our Model:} 1.0000\% accuracy
    \item \textbf{Improvement:} Significant outperformance of baselines
\end{itemize}

\section{Deployment}
\subsection{Business Criteria Assessment}
"""
Comparison with Business Success Criteria:

1. Business Objective: Identify dissatisfied passengers
   - Target: >70\% recall for dissatisfied class
   - Achieved: 100\% recall
   - Status: EXCEEDS TARGET

2. Business Objective: Enable targeted improvements
   - Target: >80\% precision
   - Achieved: 100\% precision
   - Status: EXCEEDS TARGET

3. Business Objective: Gain service insights
   - Tar...

\subsection{Deployment Recommendations}
\begin{itemize}
    \item \textbf{Hybrid Solution:} Combine model predictions with human review
    \item \textbf{Gradual Roll-out:} Start with business class passengers
    \item \textbf{Monitoring:} Implement comprehensive performance tracking
    \item \textbf{A/B Testing:} Validate impact before full deployment
\end{itemize}

\subsection{Ethical Considerations}
\begin{itemize}
    \item \textbf{Bias Monitoring:} Regular fairness audits required
    \item \textbf{Privacy Protection:} Passenger data must be anonymized
    \item \textbf{Transparency:} Model decisions should be explainable
    \item \textbf{Human Oversight:} Critical decisions require human review
\end{itemize}

\subsection{Monitoring Framework}
Key monitoring metrics include:
\begin{itemize}
    \item \textbf{Daily:} Accuracy drift (>5\% drop triggers review)
    \item \textbf{Weekly:} Fairness metrics (gender/age parity)
    \item \textbf{Monthly:} Business impact (satisfaction scores)
    \item \textbf{Immediate Intervention:} >15\% accuracy drop
\end{itemize}

\section{Conclusion}
This experiment successfully implemented a complete CRISP-DM process for predicting airline passenger satisfaction. Key achievements include:

\begin{itemize}
    \item Development of a Random Forest model with 100\% test accuracy
    \item Comprehensive bias evaluation showing minimal gender bias
    \item Full documentation of the analytics pipeline in a knowledge graph
    \item Clear deployment recommendations with monitoring framework
\end{itemize}

While performance metrics are exceptionally high, potential overfitting concerns suggest the need for further validation in real-world deployment. The hybrid deployment approach with continuous monitoring provides a responsible path forward.

\section*{Acknowledgments}
This work was conducted as part of the Business Intelligence course at TU Wien, utilizing provenance documentation through the Starvers knowledge graph system.

\end{document}
